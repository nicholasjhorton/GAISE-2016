\chapter{Appendix A: Examples of Activities and Projects}
\vspace{-.53in}
   \noindent\color{graylight}\rule[0cm]{3.25in}{0.03cm} \\
    \noindent\color{graylight}\rule[0.4cm]{3.25in}{0.03cm} \\
\color{black}
\vspace{-.25in}

\renewcommand{\labelitemi}{$\filledsquare$}

\section{\textbf{Desirable Characteristics of Class Activities}}
\begin{itemize}[leftmargin=1cm, itemsep=.2em]
\item The activity should mimic a real-world situation. It should not seem like ``busy work.'' For instance, if you use coins or cards to conduct a binomial experiment, explain real-world binomial experiments they could represent. 
\item The class should be involved in some of the decisions about how to conduct the activity. Students don't learn much from following a detailed ``recipe'' of steps.
\item The decisions made by the class should require knowledge learned in the class. For instance, if they are designing an experiment, they should consider principles of good experimental design learned in class, rather than ``intuitively'' deciding how to conduct the experiment.
\item If possible, the activity should include design, data collection, and analysis so students can see the whole process at work. 
\item It is sometimes better to have students work in teams to discuss how to design the activity and then reconvene the class to discuss how it will be done, but it is sometimes better to have the class work together for the initial design and other decisions. It depends on how difficult the issues to be discussed are and whether each team will need to do things in exactly the same way.
\item The activity should begin and end with an overview of what is being done and why.
\item The activity should be fun!
\end{itemize}
\pagebreak

\section{\textbf{Activities That Could Be Improved}}

\subsection{\textbf{\textit{Pepsi vs. Coke Activity}}}
Today, we will test whether Pepsi or Coke tastes better.\marginnote{\textit{Critique: The test is not double blind. There is no reason why the experimenter can't be blind to which drink is which. The person who initially sets up the experiment could cover or remove the labels from the drink containers and call them drinks 1 and 2. The drinks could then be prepared in advance into cups labeled A and B. The order of presentation should be randomized for each taster.}}  Divide into groups of four. Choose one person in your group to be the experimenter. \textit{Note:} If you are not the experimenter, please refrain from looking at the front of the classroom.
 
 \renewcommand{\labelenumi}{\arabic{enumi}.}
  
\begin{enumerate}[leftmargin=1cm, itemsep=.2em]
\item On the table in the front of the classroom are two large soda bottles, one of Pepsi and one of Coke. There are also cups labeled A and B. The experimenter should go to the table and flip a coin. If it's heads, then pour Pepsi into a cup labeled A and Coke into a cup labeled B. If it's tails, pour Pepsi into cup B and Coke into cup A. Remember which is which. Bring the cups back to your team.
\item Have a team member taste both drinks. Record which one he or she prefer---the one in cup A or the one in cup B.
\item The experimenter should now reveal to the team member if it was Coke or Pepsi that was preferred.
\item The experimenter should repeat this process for each team member once. Then, one of the other team members should give the taste test to the experimenter so each student will have done it once.
\item Come together as a class. Your teacher will ask how many of you preferred Coke. 
\item Look up the formula in your book for a confidence interval for a proportion. Construct a confidence interval for the proportion of students in the class who prefer Coke.
\item Do a hypothesis test for whether either drink was preferred by the class. 
\end{enumerate}



\subsection{\textbf{\textit{Central Limit Theorem Activity}}}
The purpose of this exercise is to verify the Central\marginnote{\textit{Critique: This is not a good activity for at least two reasons. First, it has absolutely no real-world motivation and reinforces the myth that statistics is boring and useless. Second, the instructions are too complete. There is no room for exploration on the part of the students; they are simply given a ``recipe'' to follow.}}
 Limit Theorem. Remember that this theorem tells us that the mean of a large sample is:
\begin{itemize}[leftmargin=1cm, itemsep=.2em]
\item Approximately bell-shaped
\item Has mean equal to the mean of the population
\item Has standard deviation equal to the population standard deviation/ sqrt(n) --- $\sigma/\sqrt{n}$
\end{itemize}

Please follow these instructions to verify that the Central Limit Theorem holds.
\begin{enumerate}[leftmargin=1cm, itemsep=.2em]
\item Divide into pairs. Each pair should have 1 die. 
\item Take turns rolling the die, 25 times each, so you will have 50 rolls. Keep track of the number that lands face up each time.
\item Draw a histogram of the results. The die faces are equally likely, so the histogram should have a ``uniform'' shape. Verify that it does.
\item Find the mean and standard deviation for the 50 rolls.
\item The mean and standard deviation for rolling a single die are 3.5 and 1.708, respectively. Is the mean for your 50 rolls close to 3.5? Is the standard deviation close to 1.708?
\item Come together as a class. Draw the theoretical curve that the mean of 50 rolls should have. Remember that it's bell-shaped and has a mean equal to the population mean, so that's 3.5 in this case, and the standard deviation in this case should be $1.708/\sqrt{50} = .24$.
\item Have each pair mark their mean for the 50 rolls on the curve. Notice whether they seem reasonable, given what is expected using the Central Limit Theorem.
\end{enumerate}

\vspace{.25in}

\noindent\allcaps{\textbf{How to improve on this activity?}} \\ 
\vspace{.1in}
The ``Cents and the Central Limit Theorem'' activity from \textit{Activity-Based Statistics} (Scheaffer et al.) provides an example for illustrating the Central Limit Theorem that is more aligned with the guidelines. Some other good examples from \textit{Activity-Based Statistics}:
\begin{itemize}[leftmargin=1cm, itemsep=.2em]
\item The introduction to hypothesis testing activity (where you draw cards at random from a deck and always get the same color) works well.
\item Matching Graphs to Variables generates a lot of discussion and learning.
\item Random Rectangles has become a standard, for good reason.
\item Randomized Response is not central to the introductory course, but it does involve some statistical thinking.
\end{itemize}

\section{\textbf{Additional Examples of Activities and Projects}}

\subsection{\textbf{\textit{Data Gathering and Analysis:  A Class of Projects}}}
The idea for projects such as the ones described here comes from Robert Wardrop's \textit{Statistics:  Learning in the Presence of Variability} (Dubuque, IA:  William C. Brown, 1995). These projects, in turn, are based on a study by cognitive psychologists Daniel Kahneman and Amos Tversky.

Consider two versions of the ``General's Dilemma'':

\begin{quotation}
Version 1:  Threatened by a superior enemy force, the general faces a dilemma. His intelligence officers say his soldiers will be caught in an ambush in which 600 of them will die unless he leads them to safety by one of two available routes.  If he takes the first route, 200 soldiers will be saved.  If he takes the second, there is a two-thirds chance that 600 soldiers will be saved and a two-thirds chance that none will be saved.  Which route should he take?
\end{quotation}

\begin{quotation}
Version 2:  Threatened by a superior enemy force, the general faces a dilemma.  His intelligence officers say his soldiers will be caught in an ambush in which 600 of them will die unless he leads them to safety by one of two available routes.  If he takes the first route, 400 soldiers will die.  If he takes the second, there is a one-third chance that no soldiers will die and a two-thirds chance that 600 will die.  Which route should he take?
\end{quotation}

Both versions of the question have the same two answers; both describe the same situation.  The two questions differ only in their wording:  One speaks of lives lost, the other of lives saved.  

A pair of questions of this form leads easily to a simple randomized comparative experiment with the two questions as ``treatments'':  Recruit a set of subjects, sort them into two groups using a random number table, and assign one version of the question to each group.  The results can be summarized in a 2x2 table of counts:

\begin{table}[!ht]
\begin{center}
\begin{tabular}{l|cc}
& \multicolumn{2}{c}{\textbf{Answer}}\\
\textbf{Question}  & A & B \\
  \hline
Version 1 &  &  \\
Version 2 &  &  \\
\end{tabular}
\end{center}
\end{table}

The data can be analyzed by comparing the two proportions. Using Fisher's exact test or the chi-square test with continuity correction, for example.

\newpage

Exercise Set 1.2 in Wardrop's book lists a large number of variations on this structure, many of them carried out by students.  Here are abbreviated versions of just four:

\renewcommand{\labelitemi}{$\closedsucc$}
\begin{itemize}[leftmargin=1cm, itemsep=.2em]
\item Ask people in a history library whether they find a particular argument from a history book persuasive; the argument was presented with and without a table of supporting data.
\item Ask women at the student union whether they would accept if approached by a male stranger and invited to have a drink; the male was/was not described as ``attractive.''
\item Ask customers ordering an ice cream cone whether they want a regular or waffle cone; the waffle cone was/was not described as ``homemade.''
\item Ask college students either (1) Would you recommend the counseling service for a friend who was depressed? Or (2) Would you go to the counseling service if you were depressed?
\end{itemize}


Projects based on two versions of a two-answer question offer a number of advantages:  
\renewcommand{\labelitemi}{$\filledsquare$}
\begin{itemize}[leftmargin=1cm, itemsep=.2em]
\item Data collection can be completed in a reasonable length of time.
\item Randomization ensures that the results will be suitable for formal inference.
\item Randomization makes explicit the connection between chance in data gathering and the use of a probability model for analysis.
\item The method of analysis is comparatively simple and straightforward.
\item The structure (a 2x2 table of counts) is one with very broad applicability.
\item Finally, the format is very open-ended, which affords students a wide range of areas of application from which to choose and offers substantial opportunities for imagination and originality in choosing subjects and the pair of questions.
\end{itemize}

\newpage

\subsection{\textbf{\textit{Team-Constructed Questions About Relationships}}}
\marginnote{Adapted from Project 2.2, Instructors' Resource Manual, \textit{Mind On Statistics}, Utts and Heckard}

\textit{These instructions are for the teacher. Instructions for students are on the Project 4 Team Form.}\\
\vspace{10pt}
\noindent\underline{Goal:} Provide students with experience in formulating a research question, then collecting and describing data to help answer it\\
\vspace{10pt}

\noindent\underline{Supplies:} (N = number of students; T = number of teams)
\begin{itemize}[leftmargin=1cm, itemsep=.2em]
\item N index cards or slips of paper of each of T colors (or use board space; see below)
\item T or 2T overhead transparencies and pens (see Step 3 for the reason for 2T of them)
\item T calculators
\end{itemize}

Students should work in teams of 4 to 6. See the Sample Project 4 Team Form.

\renewcommand{\labelenumi}{\textbf{Step \arabic{enumi}:}}
\begin{enumerate}[leftmargin=*, itemsep=1em]
\item Each team formulates two categorical variables for which they want to know if there is a relationship, \marginnote{NOTE: This can also be done with one categorical and one quantitative variable and the data retained for use when doing two-sample inference.}such as whether someone is a firstborn (or only) child and whether they prefer indoor or outdoor activities (recent research suggests that firstborns prefer indoor activities and later births prefer outdoor activities); male/female and opinion on something; class (senior, junior, etc.), and whether they own a car, etc. To make it easier to finish in time, you may want to restrict them to two categories per variable.

There are two possible methods for collecting data---using index cards (or paper) or using the board. Each of the next few steps will be described for both methods.

\item Cards: Each team is assigned a color from the T colors of index cards. For instance, Team 1 might be blue, Team 2 is pink, and so on. Board: Assign each team space on the chalkboard to write their questions.

\item Each team asks the whole class its two questions. Cards: The team writes the questions on an overhead transparency and displays them, with each team taking a turn to go to the front of the room. Students write their answers on the index card corresponding to that team's color and the team collects them. For instance, all students in the class write their answers to Team 1's questions on the blue index card, their answers to Team 2's questions on the pink card, and so on. Board: A team member writes the questions on the board along with a two-way table where each student can put a hash mark in the appropriate cell. 

\item Cards: After each team has asked its questions and students have written their answers, the cards are collected and given to the appropriate team. For instance, Team 1 receives all the blue cards. Board: All class members go to each segment of the board and put a hash mark in the cell of the table that fits them.

\item Each team tallies, summarizes, and prepares a graphical display of the data for their questions. The results are written on an overhead transparency.

\item Each team presents the results to the class.

\item Results can be retained for use when covering chi-square tests for independence if you are willing to pretend that the data are a random sample from a larger population.
\end{enumerate}




\pagebreak
\begin{flushleft} 
\textbf{\allcaps{PROJECT 4: TEAM FORM}} \\ 
 \end{flushleft}
 
 \vspace{12pt}
\noindent TEAM MEMBERS:\\ 
\vspace{12pt}
\begin{tabular}{p{2in}p{1in}p{2in}}
1. \hrulefill & & 4. \hrulefill  \\
2. \hrulefill & & 5. \hrulefill  \\
3. \hrulefill & & 6. \hrulefill  \\ \\ \\
\end{tabular}

\renewcommand{\labelenumi}{\arabic{enumi}.}

\noindent INSTRUCTIONS:
\begin{enumerate}[leftmargin=1cm, itemsep=.2em]
\item Create two categorical variables for which you think there might be an interesting relationship for class members. If you prefer, you may turn a quantitative variable into a categorical one, such as GPA---high or low (using a cut-off such as GPA $\geq$ 3.0). Each variable should have two categories to make it easier to finish in the allotted time.
\item List the two variables below, designating which is the explanatory variable and which is the response variable, if that makes sense for your situation. 

\renewcommand{\labelitemi}{$\filledsquare$}
	\begin{itemize}[leftmargin=1cm, itemsep=.2em]
	\item Explantory variable:
	\vspace{.05\textheight}
	\item Response variable:
	\vspace{.05\textheight}
	\end{itemize}
\item Each team will be assigned one segment of the chalk board. One team member is to go to the board and write your two questions. Also, write a ``two-way'' table on the board in which people will put a ``hash mark'' into the square that describes them.
\item Everyone will now go to the board and fill in a hash mark in the appropriate box for each team's set of questions. 
\item After everyone has gone to the board and filled in all their data, enter the totals in the table below for your team's questions. Also enter what the categories are for each variable. \\ 

\begin{table}
\begin{center}
\begin{tabular}{|l|c|c|c|}
\multicolumn{4}{c}{\hspace{1.8in} Response Variable}\\
\hline
\textbf{Explanatory Variable} & Category 1 & Category 2 & Total\\[.2em]
  \hline
\textbf{Category 1} & & & \\[.6em]
\hline
\textbf{Category 2} & & & \\[.6em]
\hline
Total & & & \\[.6em]
\hline
\end{tabular}
\end{center}
\end{table}
\newpage


\item Create appropriate numerical and graphical summaries to display on your team's overhead transparency. Write a brief summary of your findings below and on the back if needed. 
\item A member of each team will present the team's result to the class, using the overhead transparency.
\item Turn in this sheet and the overhead transparency.
\end{enumerate}

\newpage

\subsection{\textbf{\textit{Comparing Manual Dexterity Under Two Conditions}}}
\marginnote{Adapted from Project 12.2, Instructors' Resource Manual, \textit{Mind On Statistics}, Utts and Heckard}

\textit{These instructions are for the teacher. Instructions for students are on the ``Project 5 Team Form.''}\\
\vspace{10pt}

\noindent\underline{Goal}: Provide students with experience in designing, conducting, and analyzing an experiment\\

\vspace{10pt}

\noindent\underline{Supplies:} (N = number of students, T = number of teams)
\marginnote{NOTE: A variation is to have them do the task with and without wearing a latex glove instead of with the dominant and nondominant hand. In that case, you will need \textit{N pairs of latex gloves}}
\begin{itemize}[leftmargin=1cm, itemsep=.2em]
\item T bowls filled with about 30 of each of two distinct colors of dried beans
\item 2T empty paper cups or bowls
\item T stop watches or watches with a second hand
\end{itemize}

\noindent\underline{The Story:} A company has many workers whose job is to sort two types of small parts. Workers are prone to get repetitive strain injury, so the company wonders if there would be a big loss in productivity if the workers switch hands, sometimes using their dominant hand and sometimes using their nondominant hand. (Or if you are using latex gloves, the story can be that, for health reasons, they might want to require gloves.) Therefore, you are going to design, conduct, and analyze an experiment making this comparison. Students will be timed to see how long it takes to separate the two colors of beans by moving them from the bowl into the two paper cups, with one color in each cup. A comparison will be done after using dominant and nondominant hands. An alternative is to time students for a fixed time, such as 30 seconds, and see how many beans can be moved in that amount of time.

\renewcommand{\labelenumi}{\textbf{Step \arabic{enumi}:}}
\begin{enumerate}[leftmargin=*, itemsep=.2em]
\item \textbf{As a class, discuss how the experiment will be done. This could be done in teams first. See below for suggestions.}

	\renewcommand{\labelitemi}{$\filledsquare$}
 	 \renewcommand{\labelitemii}{$\closedsucc$}
	\begin{itemize}  
	\item What are the treatments? What are the experimental units?
	\item Principles of experimental design to consider are as follows. Use as many of them as possible in designing and conducting this experiment. Discuss why each one is used.
		\begin{itemize}[leftmargin=1cm, itemsep=.2em]
		\item Blocking or creating matched-pairs
		\item Randomization of treatments to experimental units, or randomization of order of treatments
		\item Blinding or double blinding
		\item Control group
		\item Placebo
		\item Learning effect or getting tired
		\end{itemize}
	\item What is the parameter of interest?
	\item What type of analysis is appropriate---hypothesis test, confidence interval, or both?
	\end{itemize}
\renewcommand{\labelenumi}{\textbf{Step \arabic{enumi}:}}

The class should decide that each student will complete the task once with each hand. Why is this preferable to randomly assigning half of the class to use their dominant hand and the other half to use their nondominant hand? How will the order be decided? Should it be the same for all students? Will practice be allowed? Is it possible to use a single or double-blind procedure?

\item \textbf{Divide into teams and carry out the experiment.} \\
The Project 5 Team Form shows one way to assign tasks to team members. 

\item \textbf{Descriptive statistics and preparation for inference.} \\
Convene the class and create a stemplot of the differences. Discuss whether the necessary conditions for this analysis are met. Were there any outliers? If so, can they be explained? Have someone compute the mean and standard deviation for the differences.

\item \textbf{Inference.}\\
Have teams reconvene. Each team is to find a confidence interval for the mean difference and conduct the hypothesis test.

\item \textbf{Reconvene the class and discuss conclusions.}\\
\end{enumerate}
\vspace{12pt}

\noindent\allcaps{\textbf{Suggestions for How to Design and}}\\ \noindent\allcaps{\textbf{Analyze the Experiment in Sample}} \\ \noindent\allcaps{\textbf{Project 5}}\\
\vspace{5pt}

\noindent\textbf{Design issues:}
\renewcommand{\labelitemi}{$\closedsucc$}
\begin{itemize}  [leftmargin=1cm, itemsep=.2em]
\item Blocking or creating matched-pairs: Each student should be used as a matched-pair, doing the task once with each hand.
\item Randomization of treatments to experimental units, or randomization of order of treatments: Randomize the order of which hand to use for each student.
\item Blinding or double blinding: Obviously, the student knows which hand is being used, but the time-keeper doesn't need to know.
\item Control group: Not relevant for this experiment.
\item Placebo: Not relevant for this experiment.
\item Learning effect or getting tired: There is likely to be a learning effect, so you may want to build in a few practice rounds. Also, randomizing the order of the two hands for each student will help with this. \\
\end{itemize}

\noindent \textbf{One possible design:} \\
Have each student flip a coin. Heads, start with dominant hand. Tails, start with nondominant hand. Time them to see how long it takes to separate the beans. The person timing them could be blind to the condition by not watching. \\

\vspace{20pt}
\noindent \textbf{Analysis:}\\
\vspace{8pt}
\noindent $\bullet$ \smallcaps{What is the parameter of interest?} $\bullet$ \\
\vspace{5pt}

Answer: Define the random variable of interest for each person to be a ``manual dexterity difference'' of 
\begin{equation*}
\begin{split}
d &= \textrm{number of extra seconds required with nondominant hand} \\
d   &= \textrm{time with non-dominant hand} - \textrm{time with dominant hand}\\
\end{split}
\end{equation*}

Define $\mu_d =$ population mean manual dexterity difference.\\

\vspace{18pt}
\noindent $\bullet$ \smallcaps{What are the null and alternative hypotheses?} $\bullet$
\begin{equation*}
\begin{split}
H_0 : \mu_d &= 0\\
H_A: \mu_d &> 0  \; \textrm{(faster with dominant hand)}\\
\end{split}
\end{equation*}
 
 \vspace{5pt}
\noindent $\bullet$ \smallcaps{Is a confidence interval appropriate?} $\bullet$ \\
\vspace{5pt}
Yes, it will provide information about how much faster workers can accomplish the task with their dominant hands. The formula for the confidence interval is 

\begin{equation*}
\bar{d} \pm t*\left(\frac{s_d}{\sqrt{n}}\right)
\end{equation*}
 
where $t*$ is the critical \textit{t}-value with $df = n-1$ and $s_d$ is the standard deviation of the difference scores. 
To carry out the test, compute $t=\frac{\bar{d}-0}{s_d/\sqrt{n}}$, then compare to the critical \textit{t}-value to find the \textit{p}-value. 

\pagebreak
\begin{flushleft}
\textbf{\allcaps{PROJECT 5: TEAM FORM}} \\
\end{flushleft}

\noindent TEAM MEMBERS:\\ 
\vspace{10pt}
\begin{tabular}{p{2in}p{1in}p{2in}}
1. \hrulefill & & 4. \hrulefill  \\
2. \hrulefill & & 5. \hrulefill  \\
3. \hrulefill & & 6. \hrulefill  \\ \\ \\
\end{tabular}

\noindent INSTRUCTIONS:\\ 
You will work in teams. Each team should take a bowl of beans and two empty cups. You are each going to separate the beans by moving them from the bowl to the empty cups, with one color to each cup. You will be timed to see how long it takes. You will each do this twice, once with each hand, with order randomly determined.

\renewcommand{\labelenumi}{\arabic{enumi}.}
\begin{enumerate} [leftmargin=1cm, itemsep=.2em]
\item Designate these jobs. You can trade jobs for each round if you wish.
	\renewcommand{\labelitemi}{$\filledsquare$}
	\begin{itemize} [leftmargin=1cm, itemsep=.2em]
	\item \textit{Coordinator} --- runs the show
	\item \textit{Randomizer} --- flips a coin to determine which hand each person will start with, separately for each person
	\item \textit{Time-keeper} --- must have watch with second hand to time each person for the task
	\item \textit{Recorder} --- records the results in the table below
	\end{itemize}
\item Choose who will go first. The randomizer tells the person which hand to use first. Each person should complete the task once before moving to the second hand for the first person. That gives everyone a chance to rest between hands.
\item The time-keeper times the person while they move the beans one at a time from the bowl to the cups, separating colors. 
\item The recorder notes the time and records it in the table.
\item Repeat this for each team member.
\item Each person then goes a second time, with the hand not used the first time.
\item Calculate the difference for each person.
\end{enumerate}
\pagebreak

\noindent \allcaps{\textbf{RESULTS FOR THE CLASS}}\\
\vspace{14pt}
\noindent Record the data here:\\

\begin{table}[!ht]
\begin{center}
\begin{tabular}{|l|c|c|c|}
\hline
NAME \hspace{1.5in} & Time for \textit{non} & Time for & \textit{d} = difference\\
& \textit{dominant} hand & \textit{dominant} hand & nondominant $-$ dominant hand\\[.2em]
  \hline
 & & &\\[.6em]
\hline
 & & &\\[.6em]
\hline
 & & &\\[.6em]
\hline
 & & &\\[.6em]
\hline
 & & &\\[.6em]
\hline
 & & &\\[.6em]
\hline
\end{tabular}
\end{center}
\end{table}

\noindent Parameter to be tested and estimated is:\\

\vspace{.25\textheight}

\noindent Confidence interval:\\

\vspace{.25\textheight}

\noindent Hypothesis test---hypotheses and results:


