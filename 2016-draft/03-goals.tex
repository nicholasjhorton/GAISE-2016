\chapter{Goals for Students in an Introductory Course: What it Means to be Statistically Educated}
\vspace{-.53in}
   \noindent\color{graylight}\rule[0cm]{3.25in}{0.03cm} \\
    \noindent\color{graylight}\rule[0.4cm]{3.25in}{0.03cm} \\
\color{black}
\vspace{.05in}

\newthought{Some people teach courses} that are heavily slanted toward teaching students to become statistically literate and wise consumers of data; this is somewhat similar to an art appreciation course. Some teach courses more heavily slanted toward teaching students to become producers of statistical analyses; this is analogous to the studio art course. Most courses are a blend of consumer and producer components, but the balance of that mix will determine the importance of each recommendation we present.   

The desired result of all introductory statistics courses is to produce statistically educated students, which means that students should develop statistical literacy and the ability to think statistically. The following goals represent what such a student should know and understand. Achieving this knowledge will require learning some statistical techniques, but the specific techniques are not as important as the knowledge that comes from going through the process of learning them. Therefore, we are not recommending specific topical coverage.\\
\vspace{12pt}

\noindent\textbf{Students should believe and understand why:}

\begin{itemize}[leftmargin=1cm, itemsep=.2em]
\renewcommand\labelitemi{$\closedsucc$}
\item Data beat anecdotes
\item Variability is natural, predictable, and quantifiable
\item Random sampling allows results of surveys and experiments to be extended to the population from which the sample was taken
\item Random assignment in comparative experiments allows cause-and-effect conclusions to be drawn
\item Association is not causation
\item Statistical significance does not necessarily imply practical importance, especially for studies with large sample sizes
\item Finding no statistically significant difference or relationship does not necessarily mean there is no difference or no relationship in the population, especially for studies with small sample sizes
\end{itemize}

\renewcommand\labelitemi{$\closedsucc$}
\noindent\textbf{Students should recognize:}
\begin{itemize}[leftmargin=1cm, itemsep=.2em]
\item Common sources of bias in surveys and experiments
\item How to determine the population to which the results of statistical inference can be extended, if any, based on how the data were collected
\item How to determine when a cause-and-effect inference can be drawn from an association based on how the data were collected (e.g., the design of the study)
\item That words such as ``normal," ``random,'' and ``correlation'' have specific meanings in statistics that may differ from common usage
\end{itemize}

\noindent\textbf{Students should understand the parts of the process through which statistics works to answer questions, namely:}
\begin{itemize}[leftmargin=1cm, itemsep=.2em]
\item How to obtain or generate data
\item How to graph the data as a first step in analyzing data, and how to know when that's enough to answer the question of interest
\item How to interpret numerical summaries and graphical displays of data---both to answer questions and to check conditions (to use statistical procedures correctly)
\item How to make appropriate use of statistical inference
\item How to communicate the results of a statistical analysis 
\end{itemize}

\noindent\textbf{Students should understand the basic ideas of statistical inference, including:}
\begin{itemize}[leftmargin=1cm, itemsep=.2em]
\item The concept of a sampling distribution and how it applies to making statistical inferences based on samples of data (including the idea of standard error)
\item The concept of statistical significance, including significance levels and \textit{p}-values
\item The concept of confidence interval, including the interpretation of confidence level and margin of error
\end{itemize}

\newpage

\noindent\textbf{Finally, students should know:}
\begin{itemize}[leftmargin=1cm, itemsep=.2em]
\item How to interpret statistical results in context
\item How to critique news stories and journal articles that include statistical information, including identifying what's missing in the presentation and the flaws in the studies or methods used to generate the information
\item When to call for help from a statistician
\end{itemize}

