\chapter{Appendix 0: Evolution of the Introductory Statistics Course and
Development of Statistics Education Resources}
\vspace{-.53in}
   \noindent\color{graylight}\rule[0cm]{3.25in}{0.03cm} \\
    \noindent\color{graylight}\rule[0.4cm]{3.25in}{0.03cm} \\
\color{black}








\newthought{The modern introductory statistics course} has roots that go back a long way, to early books about statistical methods. R. A. Fisher's Statistical Methods for Research Workers, which first appeared in 1925, was aimed at practicing scientists. A dozen years later, the first edition of George Snedecor's Statistical Methods presented an expanded version of the same content, but there was a shift in audience to prospective scientists who were still completing their degrees.  By 1961, with the publication of Probability with Statistical Applications by Fred Mosteller, Robert Rourke, and George Thomas, statistics had begun to make its way into the broader academic curriculum, but statistics still had to lean heavily on probability for its legitimacy.

During the late 1960s and early 1970s, John Tukey's ideas of exploratory data analysis launched the ``data revolution" in the beginning statistics curriculum, freeing certain kinds of data analysis from ties to probability-based models.  Analysis of data began to acquire status as an independent intellectual activity that did not require hours chained to a bulky mechanical calculator. Computers later expanded the types of analysis that could be completed by learners.

Two influential books appeared in 1978: \emph{Statistics}, by David Freedman, Robert Pisani, and Roger Purves, and \emph{Statistics: Concepts and Controversies}, by David S. Moore. The publication of these two books marked the birth of what we regard as the modern introductory statistics course.

The evolution of content has been paralleled by other trends. One of these is a striking and sustained growth in enrollments. Statistics from three groups of students illustrate the growth:
\begin{itemize}
\item At two-year colleges, according to the Conference Board of the Mathematical Sciences , statistics enrollments grew from 27\% the size of calculus enrollments in 1970 to 74\% in 2000 and exceeded calculus by 2010.
\item Also from the CBMS survey, enrollments in elementary statistics courses at four-year institutions were up 56\% in math departments and 50\% in statistics departments from 2005 to 2010.  
\item The Advanced Placement exam in statistics was first offered in 1997 when 7,500 students took it, more than in the first offering of an AP exam in any subject at that time. More than four times as many students were taking the exam in 2005.  A decade later, there were over 195,000 students taking the test.
\end{itemize}

The democratization of introductory statistics has broadened and diversified the backgrounds, interests, and motivations of those who take the course. Statistics is no longer reserved for future scientists in narrow fields but is now a family of courses, taught to students at many levels, from pre-high school to post-baccalaureate, with very diverse interests and goals. A teacher of today's beginning statistics courses can no longer assume that students are quantitatively skilled and adequately motivated by their career plans. 

Not only have the ``what, why, who, and when" of introductory statistics been changing, but so has the ``how." The last few decades have seen an extraordinary level of activity focused on how students learn statistics and on how teachers can effectively help them learn.


\subsection{Influential Documents on the Teaching of Statistics}

As part of the Curriculum Action Project of the Mathematics Association of America (MAA), George Cobb coordinated a focus group about important issues in statistics education. The 1992 report was published in the MAA volume Heeding the Call for Change. It included the following recommendations:

Emphasize Statistical Thinking
Any introductory course should take as its main goal helping students to learn the basic elements of statistical thinking. Many advanced courses would be improved by a more explicit emphasis on those same basic elements, namely:
\begin{itemize}
\item The need for data. The importance of data production. 
\item The omnipresence of variability. 
\item The quantification and explanation of variability. 
\end{itemize}

\subsection{More Data and Concepts, Less Theory and Fewer Recipes}

Almost any course in statistics can be improved by more emphasis on data and concepts, at the expense of less theory and fewer recipes.  To the maximum extent feasible, calculations and creation of graphics should be automated or facilitated using software.

\subsection{Foster Active Learning}
As a rule, teachers of statistics should rely much less on lecturing and much more on alternatives such as projects, lab exercises, and group problem-solving and discussion activities. Even within the traditional lecture setting, it is possible to get students more actively involved.

The three recommendations were intended to apply quite broadly (e.g., whether or not a course has a calculus prerequisite and regardless of the extent to which students are expected to learn specific statistical methods). Cobb’s focus group evolved into the joint ASA/MAA Committee on Undergraduate Statistics.  A growing body of statistics educators were implementing the recommendations and actively sharing their experiences with peers, often through projects funded by the National Science Foundation (NSF).

In the late 1990s, Joan Garfield led an NSF-funded survey  to explore the impact of this educational reform movement.  A large number of statistics instructors from mathematics and statistics departments and a smaller number of statistics instructors from departments of psychology, sociology, business, and economics were included. The responses were encouraging: many reported increased use of technology, diversification of assessment methods, and successful implementation of active learning strategies.

The American Statistical Association funded a strategic initiative to create a set of Guidelines for Assessment and Instruction in Statistics Education (GAISE) at the outset of the 21st century.  This was a two-part project that resulted in the publication  of A Pre-K–12 Curriculum Framework  and the original 2005 College Report that expanded upon the recommendations from the Cobb Report to address technology and assessment. These two reports have had a profound effect on the practice and training of educators at all levels.

Since the GAISE publications, the widespread adoption of the Common Core State Standards  has both strengthened the status of statistics as an academic necessity and challenged the content of a first collegiate course in statistics.  Just as the 1970s brought students into the statistics class who did not have strong quantitative skills but new technology allowed a shift away from heavy computation, the 2020s may bring students to our classrooms with knowledge of content that has been part of an introductory course for a long time which may be leveraged to allow new levels of complexity to be included.

\subsection{The Emergence of Statistics Education Research and Resources}

Even before the publication of the original GAISE College Report, distinctions between mathematics education and statistics education were being made.  The connection, however, remains important and interesting to both mathematicians and statisticians.  The American Statistical Association (ASA) maintains joint committees with the Mathematical Association of America (MAA), the American Mathematical Association of Two-Year Colleges (AMATYC), and the National Council of Teachers of Mathematics (NCTM).

The Statistical Education Section is one of the oldest sections within the ASA, founded in 1948, originally focused on the education of professional statisticians.  The current mission statement includes advising the Association on educational elements in communication with non-statistical audiences, promoting reach and practice in statistical education; supporting the dissemination of development/funding opportunities, teaching resources, and research findings in statistical education; and, improving the pipeline from K-12 through colleges to statistics professionals. 

In 2014 the ASA and MAA jointly endorsed a set of guidelines  for those teaching the introductory statistics course.

Instructors ideally would have the following qualifications:
\begin{itemize}
\item Experience with data and appropriate use of technology to support data analyses
\item Deep knowledge of statistics and appreciation for the differences between statistical thinking and mathematical thinking
\item Understanding the ways statisticians work with real data and approach problems and experiencing the joys of making discoveries using statistical reasoning
\item Mentoring by an experienced statistics instructor for an instructor unfamiliar with the data-driven techniques used in modern introductory statistics courses
\end{itemize}

Minimum suggested qualifications to teach would include:
\begin{itemize}
\item Two statistical methods courses, including content knowledge of data-collection methods, study design, and statistical inference
\item Experience with data analysis beyond material taught in the introductory class (e.g., advanced courses, projects, consulting, or research)
\end{itemize}

In 2000, the MAA founded a special interest group (SIGMAA) on Statistics Education.  Their purpose is also four-fold:  facilitate the exchange of ideas about teaching statistics, the undergraduate statistics curriculum, and other issues related to providing effective/engaging encounters for students; foster increased understanding of statistics through publication; promote the discipline of statistics among students; and, work cooperatively with other organizations to encourage effective teaching and learning .

The AMATYC Committee on Statistics was founded in 2010 to provide a forum for the exchange of ideas, the sharing of resources, and the discussion of issues of interest to the statistics community. The Committee pays particular attention to activities that provide professional development and foster the use of the GAISE College report in the community college setting. It also serves as a liaison with faculty at four-year institutions and with other professional organizations for the purpose of resource sharing (see the AMATYC Statistics Resources Page ).

A 2006 charter established the Consortium for the Advancement of Undergraduate Statistics Education CAUSE which had grown out of a 2002 strategic initiative within the ASA.  The mission of CAUSE is to support and advance undergraduate statistics education through resources, professional development, outreach and research.  \url{CAUSEweb.org} serves as a repository for all of those areas.  CAUSE also coordinates the US Conference on Teaching Statistics (USCOTS) which has been held in May of odd-numbered years since 2005.  Since 2012, the electronic Conference on Teaching Statistics (eCOTS) has provided a virtual experience on the even-numbered years.

The oldest conference for statistics educators, however, is sponsored by the International Association of Statistics Educators (IASE), a section of the International Statistical Institute.  The International Conference on Teaching Statistics (ICOTS) has been held every four years since 1982 at various global locations.  The IASE also supports the Statistics Education Research Journal (SERJ), a peer-reviewed e-journal in publication since 2002.

Other refereed journals of interest to statistics educators include \emph{Teaching Statistics}, the \emph{Journal of Statistics Education}, and \emph{Technology Innovations in Statistics Education}.

XX Nick note: why not alphabetical?



XX References need to be formatted.

\begin{verbatim}
  http://www.ams.org/profession/data/cbms-survey/cbms
  http://research.collegeboard.org/programs/ap/data/participation/ap-2015
  George Cobb. Heeding the Call for Change: Suggestions for Curricular Action (MAA Notes No. 22), chapter “Teaching Statistics,” pages 3–43. The Mathematical Association of America, Washington DC, 1992
  Garfield, J. (2000). An evaluation of the impact of statistics reform: Final Report. National Science Foundation (REC-9732404).
  amstat.org/education/gaise
  Christine Franklin (2007). Guidelines for assessment and instruction in statistics education
(GAISE) report: A pre-k–12 curriculum framework. American Statistical Association: Alexandria, VA
  http://www.corestandards.org/
  Ben-Zvi, D., & Garfield, J. (2008). “Introducing the Emerging Discipline of Statistics Education,” School Science & Mathematics, 108(8), 355-361.  doi:10.1111/j.1949-8594.2008.tb17850.x
  http://community.amstat.org/statisticaleducationsection/home
  http://magazine.amstat.org/blog/2014/04/01/asamaaguidelines/
  http://sigmaa.maa.org/stat-ed/art1.html
  http://www.amatyc.org/?page=StatsResources
  http://iase-web.org/
  http://onlinelibrary.wiley.com/journal/10.1111/(ISSN)1467-9639
  http://www.amstat.org/publications/jse/
  http://escholarship.org/uc/uclastat_cts_tise
\end{verbatim}
