
%%
% Start the main matter (normal chapters)
\mainmatter
\blankpage

\chapter{Introduction}
\vspace{-.53in}
   \noindent\color{graylight}\rule[0cm]{3.25in}{0.03cm} \\
    \noindent\color{graylight}\rule[0.4cm]{3.25in}{0.03cm} \\
\color{black}
\vspace{.05in}

\newthought{The GAISE project} was funded by a member initiative grant from the ASA in 2003 to develop ASA-endorsed guidelines for assessment and instruction in statistics in the K--12 curriculum and for the introductory college statistics course.   

Our work on the college course guidelines included many discussions over email and in-person small group meetings.  Our discussions began by reviewing existing standards and guidelines, relevant research results from the studies of teaching and learning statistics, and recent discussions and recommendations regarding the need to focus instruction and assessment on the important concepts that underlie statistical reasoning.

\section{\textbf{History and Growth of the Introductory Course}}
\newthought{The modern introductory statistics course} has roots that go back a long way, to early books about statistical methods.  R. A. Fisher's \textit{Statistical Methods for Research Workers}, which first appeared in 1925, was aimed at practicing scientists.  A dozen years later, the first edition of George Snedecor's \textit{Statistical Methods} presented an expanded version of the same content, but there was a shift in audience.  More than Fisher's book, Snedecor's became a textbook used in courses for prospective scientists who were still completing their degrees; statistics was beginning to establish itself as an academic subject, albeit with heavy practical, almost vocational emphasis.  By 1961, with the publication of \textit{Probability with Statistical Applications} by Fred Mosteller, Robert Rourke, and George Thomas, statistics had begun to make its way into the broader academic curriculum, but here again, there was a catch: In these early years, statistics had to lean heavily on probability for its legitimacy.  

During the late 1960s and early 1970s, John Tukey's ideas of exploratory data analysis brought a near-revolutionary pair of changes to the curriculum: freeing certain kinds of data analysis from ties to probability-based models so that the analysis of data could begin to acquire status as an independent intellectual activity and introducing a collection of ``quick-and-dirty'' data tools so students could analyze data without having to spend hours chained to a bulky mechanical calculator.  Computers would later complete the ``data revolution'' in the beginning statistics curriculum, but Tukey's ideas of exploratory data analysis (EDA) provided both the first technical breakthrough and the new ethos that avoided invented examples.  

Two influential books appeared in 1978:  \textit{Statistics}, by David Freedman, Robert Pisani, and Roger Purves, and \textit{Statistics: Concepts and Controversies}, by David S. Moore. The publication of these two books marked the birth of what we regard as the modern introductory statistics course.  

The evolution of content has been paralleled by other trends.  One of these is a striking and sustained growth in enrollments.  Two sets of statistics suffice here:  


\renewcommand\labelitemi{$\filledsquare$}
\begin{itemize}[leftmargin=1cm]
\item At two-year colleges, according to the Conference Board of the Mathematical Sciences, statistics enrollments have grown from 27\% of the size of calculus enrollments in 1970 to 74\% of the size of calculus enrollments in 2000.
\item The Advanced Placement exam in statistics was first offered in 1997.  There were 7,500 students who took it that first year, more than in the first offering of an AP exam in any subject at that time.  The next year, more than 15,000 students took the exam.  The next year, more than 25,000, and the next, 35,000.  In 2004, more than 65,000 students took the AP statistics exam.
\end{itemize}

Both the changes in course content and the dramatic growth in enrollment are implicated in a third set of changes, a process of democratization that has broadened and diversified the backgrounds, interests, and motivations of those who take the courses.  Statistics has gone from being a course taught from a book like Snedecor's, for a narrow group of future scientists in agriculture and biology, to being a family of courses, taught to students at many levels, from pre-high school to post-baccalaureate, with very diverse interests and goals. A teacher in the 1940s, using Snedecor's \textit{Statistical Methods}, could assume that most students were both quantitatively skilled and adequately motivated by their career plans. A teacher of today's beginning statistics courses works with a different group of students.  Most take statistics earlier in their lives, increasingly often in high school; few are drawn to statistics by immediate practical need; and there is great variety in their levels of quantitative sophistication.  As a result, today's teachers face challenges of motivation and exposition that are substantially greater than those of a half century ago.

Not only have the ``what, why, who, and when'' of introductory statistics been changing, but so has the ``how.''  The last few decades have seen an extraordinary level of activity focused on how students learn statistics, and on how we teachers can be more effective in helping them learn. 


\section{\textbf{The 1992 Cobb Report}}
\newthought{In the spring of 1991}, George Cobb, in order to highlight important issues to the mathematics community, coordinated an email focus group on statistics education as part of the Curriculum Action Project of the Mathematics Association of America (MAA). The report was published in the MAA volume \textit{Heeding the Call for Change}\cite{cobb1}. It included the following recommendations:

\subsection{\textbf{Emphasize Statistical Thinking}}
Any introductory course should take as its main goal helping students to learn the basic elements of statistical thinking. Many advanced courses would be improved by a more explicit emphasis on those same basic elements, namely:

\begin{itemize}[leftmargin=1cm, itemsep=.2em]
\item \textit{The need for data}.  Recognizing the need to base personal decisions on evidence (data) and the dangers inherent in acting on assumptions not supported by evidence.
\item \textit{The importance of data production}.  Recognizing that it is difficult and time-consuming to formulate problems and to get data of good quality that really deal with the right questions. Most people don't seem to realize this until they go through this experience themselves.
\item \textit{The omnipresence of variability}.  Recognizing that variability is ubiquitous. It is the essence of statistics as a discipline and not best understood by lecture. It must be experienced.
\item \textit{The quantification and explanation of variability}. Recognizing that variability can been measured and explained, taking into consideration the following: (a) randomness and distributions; (b) patterns and deviations (fit and residual); (c) mathematical models for patterns; (d) model-data dialogue (diagnostics).
\end{itemize}


\subsection{\textbf{More Data and Concepts, Less Theory and Fewer Recipes}}
Almost any course in statistics can be improved by more emphasis on data and concepts, at the expense of less theory and fewer recipes. To the maximum extent feasible, calculations and graphics should be automated.


\subsection{\textbf{Foster Active Learning}}
As a rule, teachers of statistics should rely much less on lecturing and much more on alternatives such as projects, lab exercises, and group problem-solving and discussion activities. Even within the traditional lecture setting, it is possible to get students more actively involved.  

The three recommendations were intended to apply quite broadly (e.g., whether or not a course has a calculus prerequisite and regardless of the extent to which students are expected to learn specific statistical methods).  Although the work of the focus group ended with the completion of their report, many members of the group continued to work on these issues, especially on efforts at dissemination and implementation, as members of the joint ASA/MAA Committee on Undergraduate Statistics.

\section{\textbf{Current Status of the Introductory Statistics Course}}

\newthought{Over the decade that followed }the publication of the Cobb report, many changes were implemented in the teaching of statistics. In recent years, many statisticians have become involved in the reform movement in statistical education aimed at the teaching of introductory statistics, and the National Science Foundation has funded numerous projects designed to implement aspects of this reform\cite{cobb2}. Moore\cite{moore} describes the reform in terms of changes in content (more data analysis, less probability), pedagogy (fewer lectures, more active learning), and technology (for data analysis and simulations).   

In 1998 and 1999, Garfield\cite{garfield} surveyed a large number of statistics instructors from mathematics and statistics departments and a smaller number of statistics instructors from departments of psychology, sociology, business, and economics to determine how the introductory course is being taught and to begin to explore the impact of the educational reform movement.
 
The results of this survey suggested that major changes were being made in the introductory course, that the primary area of change was in the use of technology, and that the results of course revisions generally were positive, although they required more time of the course instructor. Results were surprisingly similar across departments, with the main differences found in the increased use of graphing calculators, active learning and alternative assessment methods in courses taught in math departments in two-year colleges, the increased use of web resources by instructors in statistics departments, and the reasons cited for why changes were made (more math instructors were influenced by recommendations from statistics education).  The results were also consistent in reporting that more changes were to be made, particularly as more technological resources became available.

Today's introductory statistics course is actually a family of courses taught across many disciplines and departments. The students enrolled in these courses have different backgrounds (e.g., in mathematics, psychology) and goals (e.g., some hope to do their own statistical analyses in research projects, some are fulfilling a general quantitative reasoning requirement). 

As in the past, some of these courses are taught in large classes and some are taught in small classes (or even freshman seminars). Some students are taught statistics in computer labs, some students take the course using only a simple calculator, and some take the course via distance learning without ever seeing their classmates or instructor in person.  Some classes are taught over a 10-week quarter and some are taught over a 15-week semester. Each of these classes might range from three to six hours per week.
 
Today's goals for students tend to focus more on conceptual understanding and attainment of statistical literacy and thinking, and less on learning a set of tools and procedures. While demands for dealing with data in an information age continue to grow, advances in technology and software make tools and procedures easier to use and more accessible to more people, thus decreasing the need to teach the mechanics of procedures, but increasing the importance of giving more people a sounder grasp of the fundamental concepts needed to use and interpret those tools intelligently. These new goals, described in the following section, reinforce the need to reexamine and revise many introductory statistics courses to help achieve the important learning goals for students.


