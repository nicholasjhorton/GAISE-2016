\chapter{Recommendations}
\vspace{-.53in}
   \noindent\color{graylight}\rule[0cm]{3.25in}{0.03cm} \\
    \noindent\color{graylight}\rule[0.4cm]{3.25in}{0.03cm} \\
\color{black}
\vspace{.05in}
 
\newthought{We endorse the ideas} in the three original goals found in the Cobb report\cite{cobb1} and have expanded them in light of today's situation. The intent of these recommendations is to help students attain the list of learning goals described in the previous section.
 
 
\section{\textbf{Recommendation 1: Emphasize statistical literacy and develop statistical thinking.}}
 
\newthought{We define statistical literacy} as understanding the basic language of statistics (e.g., knowing what statistical terms and symbols mean and being able to read statistical graphs) and fundamental ideas of statistics. For readings on statistical literacy, see Gal\cite{gal}, Rumsey\cite{rumsey}, and Utts\cite{utts}.
 
Statistical thinking has been defined as the type of thinking that statisticians use when approaching or solving statistical problems. Statistical thinking has been described as understanding the need for data, the importance of data production, the omnipresence of variability, and the quantification and explanation of variability\cite{cobb1}. We provide illustrations of statistical thinking in the following example and analogy.

\subsection{\textbf{The Funnel Example}} 
Think of a funnel that is wide at the top, corresponding to a great many situations, and narrow at the bottom, corresponding to a few specialized cases. Statisticians are practical problem-solvers.  When a client presents a problem (e.g., Is there a treatment effect present?), the statistician tries to provide a practical answer that addresses the problem efficiently. Quite often, a simple graph is sufficient to tell the story. Perhaps a more detailed plot will answer the question at hand. If not, then some calculations may be needed. A simple test based on a gross simplification of the situation may confirm that a treatment effect is present. If simplifying the situation is troublesome, then a more refined test may be used, capturing more of the specifics of the modeling situation at hand.  Different statisticians may come up with somewhat different analyses of a given set of data, but will usually agree on the main conclusions and only worry about minor points if those points matter to the client.  If there is no standard procedure to answer the question, then and only then  will the statistician use first principles to develop a new tool. \textit{We should model this type of thinking for our students, rather than showing them a set of skills and procedures and giving them the impression that, in any given situation, there is one best procedure to use and only that procedure is acceptable.}
 
\subsection{\textbf{The Carpentry Analogy}}
In week 1 of the carpentry (statistics) course, we learned to use various kinds of planes (summary statistics). In week 2, we learned to use different kinds of saws (graphs). Then, we learned about using hammers (confidence intervals). Later, we learned about the characteristics of different types of wood (tests). By the end of the course, we had covered many aspects of carpentry (statistics). But I wanted to learn how to build a table (collect and analyze data to answer a question) and I never learned how to do that. \textit{We should teach students that the practical operation of statistics is to collect and analyze data to answer questions.}

\vspace{.2in}
\noindent \allcaps{\textbf{Suggestions for teachers:}}

\renewcommand\labelitemi{$\checkmark$}

\begin{itemize}[leftmargin=1cm, itemsep=.2em]
\item Model statistical thinking for students, working examples and explaining the questions and processes involved in solving statistical problems from conception to conclusion.
\item Use technology and show students how to use technology effectively to manage data, explore data, perform inference, and check conditions that underlie inference procedures.
\item Give students practice developing and using statistical thinking.  This should include open-ended problems and projects.
\item Give students plenty of practice with choosing appropriate questions and techniques, rather than telling them which technique to use and merely having them implement it.
\item Assess and give feedback on students' statistical thinking.
\end{itemize}
 
In the appendices, we present examples of projects, activities, and assessment instruments that can be used to develop and evaluate statistical thinking.

\section{\textbf{Recommendation 2: Use real data.}}
 
\newthought{It is important to use real data} in teaching statistics to be authentic to consider issues related to how and why the data were produced or collected, and to relate the analysis to the problem context. Using real data sets of interest to students is also a good way to engage them in thinking about the data and relevant statistical concepts. There are many types of real data, including archival data, classroom-generated data, and simulated data. Sometimes, hypothetical data sets may be used to illustrate a particular point  (e.g., The Anscombe data illustrates how four data sets can have the same correlation but strikingly different scatterplots.) or to assess a specific concept. It is important to only use created or realistic data for this specific purpose and not for general data analysis and exploration. An important aspect of dealing with real data is helping students learn to formulate good questions and use data to answer them appropriately based on how the data were produced. 

\vspace{.2in}
\noindent \allcaps{\textbf{Suggestions for teachers:}}

\begin{itemize}[leftmargin=1cm,  itemsep=.2em]
\item Search for good, raw data to use from web data repositories, textbooks, software packages, and surveys or activities in class. If there is an opportunity, seek out real data directly from a practicing research scientist (through a journal or at one's home institution). Using such data can enliven your class and increase the store of good data sets for other teachers by communicating the newly found data to others.
Search for and use summaries based on real data, from data summary websites, journal articles, websites with surveys and polls, and textbooks.
\item Use data to answer questions relevant to the context and generate new questions.
\item Make sure questions used with data sets are of interest to students---if no one cares about the questions, it's not a good data set for the introductory class. (Example: physical measurements on species no one has heard of.) \textit{Note:  Few data sets interest all students, so one should use data from a variety of contexts.}
\item Use class-generated data to formulate statistical questions and plan uses for the data before developing the questionnaire and collecting the data. (Example: Ask questions likely to produce different shaped histograms, use interesting categorical variables to investigate relationships.) It is important that data gathered from students in class not contain information that could be embarrassing to students and that students' privacy is maintained.
\item Get students to practice entering raw data using a small data set or a subset of data, rather than spending time entering a large data set. Make larger data sets available electronically.
\item Use subsets of variables in different parts of the course, but integrate the same data sets throughout. (Example: Do side-by-side boxplots to compare two groups, then do two-sample \textit{t}-tests on the same data. Use histograms to investigate shape, then to verify conditions for hypothesis tests.)
\end{itemize}
 
The appendices include examples of good ways (and not-so-good ways) to use data in homework, projects, tests, etc.

\section{\textbf{Recommendation 3: Stress conceptual understanding, rather than mere knowledge of procedures.}}
 
\newthought{Many introductory courses} contain too much material, and students end up with a collection of ideas that are understood only at surface level, are not well-integrated, and are quickly forgotten. If students don't understand the important concepts, there's little value in knowing a set of procedures. If they understand the concepts well, then particular procedures will be easy to learn. In the student's mind, procedural steps too often claim attention that an effective teacher could otherwise direct toward concepts.
 
Recognize that giving more attention to concepts than to procedures may be difficult politically, both with students and client disciplines. However, students with a good conceptual foundation from an introductory course are well-prepared to study additional statistical techniques such as research methods, regression, experimental design, or statistical methods in a second course.
 
 \vspace{.2in}
\noindent \allcaps{\textbf{Suggestions for teachers:}}
\begin{itemize}[leftmargin=1cm, itemsep=.2em]
\item View the primary goal as not to cover methods, but to discover concepts.
\item Focus on students' understanding of key concepts, illustrated by a few techniques, rather than covering a multitude of techniques with minimal focus on underlying ideas.
\item Pare down content of an introductory course to focus on core concepts in more depth.  \textit{Examples of syllabi focused on concepts, compared to a syllabus focused on a list of topics, are in the appendices.}
\end{itemize}
 
Perform routine computations using technology to allow greater emphasis on interpretation of results. Although the language of mathematics provides compact expression of key ideas, use formulas that enhance the understanding of concepts, and avoid computations that are divorced from understanding.  For example, $s=\sqrt{\frac{\Sigma(y-\bar{y})^2}{n-1}}$ helps students understand the role of standard deviation as a measure of spread and to see the impact of individual \textit{y} values on \textit{s}, whereas $s=\sqrt{\frac{\Sigma y^2 - \frac{1}{n}\left(\Sigma y\right)^2}{n-1}}$ has no redeeming pedagogical value.

\section{\textbf{Recommendation 4: Foster active learning in the classroom.}}
 
\newthought{Using active learning methods in class} is a valuable way to promote collaborative learning, allowing students to learn from each other. Active learning allows students to discover, construct, and understand important statistical ideas and to model statistical thinking. Activities have an added benefit in that they often engage students in learning and make the learning process fun.  Other benefits of active learning methods are the practice students get communicating in the statistical language and learning to work in teams. Activities offer the teacher an informal method of assessing student learning and provide feedback to the instructor on how well students are learning. It is important that teachers not underestimate the ability of activities to teach the material or overestimate the value of lectures, which is why suggestions are provided for incorporating activities, even in large lecture classes.  \\
\vspace{.25in}
 
\noindent\allcaps{\textbf{Types of active learning include:}}

\renewcommand\labelitemi{$\filledsquare$}

\begin{itemize}[leftmargin=1cm, itemsep=.2em]
\item Group or individual problem-solving, activities, and discussion
\item Lab activities (physical and computer-based)
\item Demonstrations based on data generated on the spot from the students
\end{itemize}
 
\renewcommand\labelitemi{$\checkmark$}

\vspace{.2in}
\noindent \allcaps{\textbf{Suggestions for teachers:}}
\begin{itemize}[leftmargin=1cm, itemsep=.2em]
\item Ground activities in the context of real problems. Therefore, data should be collected to answer a question, not ``collect data to collect data'' (without a question).
\item Mix lectures with activities, discussions, and labs.
\item Precede computer simulations with physical explorations (e.g., die rolling, card shuffling).
\item Collect data from students (anonymously).
\item Encourage predictions from students about the results of a study that provides the data for an activity before analyzing the data. This motivates the need for statistical methods. (If all results were predictable, we wouldn't need either data or statistics.)
\item Do not use activities that lead students step by step through a list of procedures, but allow students to discuss and think about the data and the problem.  
\item Plan ahead to make sure there is enough time to explain the problem, let the students work through the problem, and wrap up the activity during the same class. It is hard to complete the activity in the next class period. Make sure there is time for recap and debriefing, even if at the beginning of the next class period.
\item Provide a lot of feedback to students on their performance and learning.
\item Include assessment as an important component of an activity.
\end{itemize}
 
\vspace{.2in}
\noindent \allcaps{\textbf{Suggestions for implementing active learning in large classes:}}
\begin{itemize}[leftmargin=1cm, itemsep=.2em]
\item Take advantage of large classes providing opportunities for large sample sizes for student-generated data.
\item In large classes, it may be easier to have students work in pairs, rather than in larger groups.
\item Use a separate lab/discussion section for activities, if possible.
\end{itemize}

\section{\textbf{Recommendation 5: Use technology for developing concepts and analyzing data.}}
 
\newthought{Technology has changed} \marginnote{See the Appendices for an example illustrating technology uses.} the way statisticians work and should change what and how we teach.  For example, statistical tables such as a normal probability table are no longer needed to find \textit{p}-values, and we can implement computer-intensive methods. We think technology should be used to analyze data, allowing students to focus on interpretation of results and testing of conditions, rather than on computational mechanics.Technology tools should also be used to help students visualize concepts and develop an understanding of abstract ideas by simulations. Some tools offer both types of uses, while, in other cases, a statistical software package may be supplemented by web applets. Regardless of the tools used, it is important to view the use of technology not just as a way to compute numbers but as a way to explore conceptual ideas and enhance student learning as well. We caution against using technology merely for the sake of using technology (e.g., entering 100 numbers in a graphing calculator and calculating statistical summaries) or for pseudo-accuracy (carrying out results to multiple decimal places). Not all technology tools will have all desired features. Moreover, new ones appear all the time.\\
\vspace{.2in}
 
\noindent\allcaps{\textbf{Technologies available:}}

\renewcommand\labelitemi{$\filledsquare$}

\begin{itemize}[leftmargin=1cm, itemsep=.2em]
\item Graphing calculators
\item Statistical packages
\item Educational software
\item Applets
\item Spreadsheets
\item Web-based resources, including data sources, online texts, and data analysis routines
\item Classroom response systems
\end{itemize}
 
 \vspace{.2in}
\noindent\allcaps{\textbf{Suggestions for teachers on ways to use technology:}}
\renewcommand\labelitemi{$\checkmark$}

\begin{itemize}[leftmargin=1cm, itemsep=.2em]
\item Access large, real data sets
\item Automate calculations
\item Generate and modify appropriate statistical graphics
\item Perform simulations to illustrate abstract concepts
\item Explore ``what happens if \ldots'' questions
\item Create reports
\end{itemize}

\vspace{.2in}
 
\noindent\allcaps{\textbf{Considerations for teachers when selecting technology tools:}}

\renewcommand\labelitemi{$\filledsquare$}

\begin{itemize}[leftmargin=1cm, itemsep=.2em]
\item Ease of data entry, ability to import data in multiple formats
\item Interactive capabilities
\item Dynamic linking between data, graphical, and numerical analyses 
\item Ease of use for particular audiences
\item Availability to students, portability\\
\end{itemize}


\section{\textbf{Recommendation 6: Use assessments to improve and evaluate student learning.}}

\newthought{Students will value what you assess}.
Therefore, assessments need to be aligned with learning goals.
\marginnote{\added{\url{CAUSEweb.org} hosts a repository of learning outcomes for introductory statistics courses.}}
Assessments need to focus on \added{demonstrating} understanding key ideas, not just on skills, procedures, and computed answers. \changed{A course should include} formative assessments (e.g., quizzes, midterm exams, and small projects) \changed{along} with summative evaluations (\added{e.g., exams and} course grades). Useful and timely feedback is essential for assessments to lead to learning.  Types of assessment may be more or less practical in different types of courses.
However, it is possible, even in large classes, to implement good assessments. \\
\marginnote{See the Appendices for examples of \changed{model} assessment items and suggestions for improving weak items.  \added{Other rich sources of items include the ARTIST project (Assessment Resource Tools for Improving Statistical Thinking, \url{https://apps3.cehd.umn.edu/artist}) and LOCUS (Levels of Conceptual Understanding in Statistics, \url{https://locus.statisticseducation.org})}.}

\vspace{.2in}

\noindent\allcaps{\textbf{Types of assessment:}}
\begin{itemize}[leftmargin=1cm, itemsep=.2em]
\item Homework
\item \added{In-class} Quizzes and exams
\item \added{Online quizzes and activities}
\item Minute papers
\item Projects
\item Activities
\item Oral presentations
\item Written reports
\item Videos reports
\item Article critiques
\end{itemize}

 \vspace{.2in}
\noindent\allcaps{\textbf{Suggestions for teachers:}}

\renewcommand{\labelitemi}{$\checkmark$}

\begin{itemize}[leftmargin=1cm, itemsep=.2em]
\item Integrate assessment as an essential component of the course. Assessment tasks that are well-coordinated with what the teacher is doing in class are more effective than tasks that focus on what happened in class two weeks earlier.
\item Use a variety of assessment methods to provide a more complete evaluation of student learning.
\item Assess statistical literacy using assessments such as interpreting or critiquing articles in the news and graphs in media.
\item Assess statistical thinking using assessments such as student projects and open-ended investigative tasks.
\end{itemize}
 
 \vspace{.2in}
\noindent\allcaps{\textbf{Suggestions for student assessment in large classes:}}
\begin{itemize}[leftmargin=1cm, itemsep=.2em]
\item Use small group projects instead of individual projects.
\item Use peer review of projects to provide feedback and improve projects before grading.
\item Use items that focus on choosing good interpretations of graphs or selecting appropriate statistical procedures.
\item Use discussion sections for student presentations.
\end{itemize}


\section{\textbf{Making It Happen}}
 
\newthought{Statistics education} has come a long way since Fisher and Snedecor. Moreover, teachers of statistics across the country have generally been enthusiastic about adopting modern methods and approaches. Nevertheless, changing the way we teach isn't always easy. In a way, we are all teachers and learners, a bit like hermit crabs:  To grow, we must first abandon the protective shell of what we are used to and endure a period of vulnerability until we can settle into a new and larger set of habits and expectations. 
 
We have presented many ideas in this report.  We advise readers to move in the directions suggested by taking small steps at first.  Examples of small steps include the following:

\renewcommand{\labelitemi}{$\filledsquare$}

\begin{itemize} [leftmargin=1cm, itemsep=.2em]
\item Adding an activity to your course
\item Having your students do a small project
\item Integrating an applet into a lecture
\item Demonstrating the use of software to your students
\item Increasing the use of real data sets
\item Deleting a topic from the list you currently try to cover to focus more on understanding concepts
\end{itemize}

Your teaching philosophy will inform your choice of textbook, but the recommendations in this report are not about choosing a text.  They are about a way of teaching. 
 
There are many resources available, including the MAA Notes volumes that deal with teaching statistics, the Consortium to Advance Undergraduate Statistics Education (CAUSE) (causeweb.org), the Isostat discussion list (\url{www.lawrence.edu/fac/jordanj/isostat.html}), the SIGMAA- Stat Ed group within the MAA (\url{www.pasles.org/sigmaastat}), and the ASA website, especially the Center for Statistics Education (\url{www.amstat.org/education}) and the Statistical Education Section (\url{www.amstat.org/sections/educ}).
